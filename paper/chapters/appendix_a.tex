\documentclass{llncs}
\usepackage[utf8]{inputenc}
\usepackage{verbatim}
\usepackage{multicol}
\usepackage{llncsdoc}
\usepackage{amsmath}
\usepackage{amsfonts}
\usepackage{amssymb}
\usepackage{graphicx}
\usepackage{lmodern}
\usepackage{calc}
\usepackage{enumitem}
\usepackage{algpseudocode}
\usepackage{algorithm}
\usepackage{algorithmicx}

\algsetblockdefx[IfContinue]{IfContinue}{IfContinue}
{0}{0pt}
[0]{}
[1]{\textbf{if} #1 \textbf{continue}}

\algrenewcommand\algorithmicrequire{%
  \makebox[\widthof{\textbf{Output:}}][l]{\textbf{Input:}}}
  
 \algrenewcommand\algorithmicensure{%
  \textbf{Output:}}

\usepackage{color}
\usepackage{gnuplottex}
\usepackage{subcaption}
\usepackage{microtype}
\usepackage[normalem]{ulem}
\captionsetup{compatibility=false}
\usepackage{tikz}
\usetikzlibrary{trees,automata,positioning}
\usepackage{booktabs}
\usepackage{gnuplottex}
\usepackage{xparse}
\usepackage{epstopdf}
% For scaling gnuplottex
\ExplSyntaxOn
\DeclareExpandableDocumentCommand{\convertlen}{ O{cm} m }
 {
  \dim_to_unit:nn { #2 } { 1 #1 } cm
 }
\ExplSyntaxOff

%% For lattice figure
% Set the overall layout of the tree
\tikzstyle{level 1}=[level distance=3.0cm, sibling distance=0.6cm]
\tikzstyle{level 2}=[level distance=3.5cm, sibling distance=0.6cm]
\tikzstyle{level 3}=[level distance=3.5cm, sibling distance=0.6cm]

% Define styles for bags and leafs
\tikzstyle{l1} = [rectangle, text width=5em, text centered]
\tikzstyle{l2} = [rectangle, text width=5em, text centered]
\tikzstyle{l3} = [rectangle, text width=5em, text centered]

% only when using asmthm
%\newtheorem{definition}{Definition}
%\newtheorem{theorem}{Theorem}

\author{Micky Faas \and Matthijs van Leeuwen}
\title{VOUW: Geometric Pattern Mining using the MDL Principle}
\institute{Leiden Institute for Advances Computer Science}
\begin{document}

\section{Appendix}
\label{appendix_a}
\subsection{Prequential Plugin-Code}

To encode the instance matrix we use the \textbf{prequential plug-in code} \cite{ppcode}. The prequential plug-in code is defined for sequences of one item at a time and updates the probability of each item as it is encoded, such that the probability need not be known in advance. It has the favorable property of being asymptotically equal to the optimal code for large sequences. Say we want to encode all elements ${I}_i \in {I}$, we define:
%\begin{definition}
\label{plugin}
\begin{align}
P_{plugin}( y_i = {I}_i \mid y^{i-1} ) = \frac{|\{y \in y^{i-1} \mid y = {I}_i\}| + \epsilon }{\sum_{X \in H}|\{y \in y^{i-1} \mid y = X\}| + \epsilon}
\end{align}
%\end{definition}
Here $y_i$ is the i-th element to be encoded and $y^{i-1}$ is the sequence of elements encoded so far. We initialize the base case (no element has been sent yet) with a pseudocount $\epsilon$, which gives $P_{plugin}( y_1 = {I} \mid y^{0} ) = \frac{\epsilon}{\epsilon|H|}$. We pick $\epsilon=0.5$ as it is used generally with good results.

Let us adapt this principle to the problem of encoding patterns. The first step here is to determine the probability that each unique element (instance of a pattern) in ${I}$ occurs. 

%\begin{definition}
\label{usage}
\smallskip \noindent $\triangleright$
\emph{Given a set of instances ${I}$, we define $\mathrm{usage}(X) = |\{ {I}_i \in {I} \mid {I}_i = X\}|.$}
\smallskip
%\end{definition}

From this definition we see that the \textbf{usage} of a pattern is a sum of how often it occurs as an instance. We can use this function to simplify things a little by realizing that we actually know the precise number of instances per pattern on the side of the decoder, but not as the decoder. This information can be used to slightly rephrase Equation \ref{plugin} to be able to encode items in arbitrary order. This produces the length function of the instance matrix ${I}$ as follows:
\footnote{Here we use the fact that we can interchange sums of logarithms with logarithms of products and that those terms can be moved around freely. Moreover we convert the real-valued product sequences to the Gamma function $\Gamma$, which is the factorial function extended to real and complex numbers such that $\Gamma(n) = (n-1)!$.}:
\begin{align}
\begin{split}
	L_{pp}({I}\mid P_{plugin}) &= \sum^{|{I}|}_{i=1} -\log \frac{|\{y \in y^{i-1} \mid y = {I}_i\}| + \epsilon }{\sum_{X \in H}|\{y \in y^{i-1} \mid y = X\}| + \epsilon}\\
	&= \sum^{|H|}_{X_i \in h} -\log \prod^{\mathrm{usage}(X_i)-1}_{j=0} \frac{j+\epsilon}{\sum^{i-1}_{k=1} U(X_k)+j+\epsilon|H|} \\
	&= -\log \frac{\prod^{X_i\in H} \prod^{\mathrm{usage}(X_i)}_{j=0} j + \epsilon}{\prod^{|{I}|-1}_{j=0} j + \epsilon|H|} \\
	&= -\sum^{|H|}_{X_i \in h} \left[ \log \frac{\Gamma(\mathrm{usage}(X_i)+\epsilon)}{\Gamma(\epsilon)}\right] + \log \frac{\Gamma(|{I}| + \epsilon|H|)}{\Gamma(\epsilon|H|)}
\end{split}
\end{align}

%Lastly, in addition to $L_{\mathbb{N}}$ and $L_{pp}$ we also define the length of the uniform distribution $L_0(n)=log(n)$. That is, when $n$ items have equal probability they all receive a code of equal length $log(n)$.


\end{document}