\documentclass{llncs}
\usepackage[utf8]{inputenc}
\usepackage{verbatim}
\usepackage{multicol}
\usepackage{llncsdoc}
\usepackage{amsmath}
\usepackage{amsfonts}
\usepackage{amssymb}
\usepackage{graphicx}
\usepackage{lmodern}
\usepackage{calc}
\usepackage{enumitem}
\usepackage{algpseudocode}
\usepackage{algorithm}
\usepackage{algorithmicx}

\algsetblockdefx[IfContinue]{IfContinue}{IfContinue}
{0}{0pt}
[0]{}
[1]{\textbf{if} #1 \textbf{continue}}

\algrenewcommand\algorithmicrequire{%
  \makebox[\widthof{\textbf{Output:}}][l]{\textbf{Input:}}}
  
 \algrenewcommand\algorithmicensure{%
  \textbf{Output:}}

\usepackage{color}
\usepackage{gnuplottex}
\usepackage{subcaption}
\usepackage{microtype}
\usepackage[normalem]{ulem}
\captionsetup{compatibility=false}
\usepackage{tikz}
\usetikzlibrary{trees,automata,positioning}
\usepackage{booktabs}
\usepackage{gnuplottex}
\usepackage{xparse}
\usepackage{epstopdf}
% For scaling gnuplottex
\ExplSyntaxOn
\DeclareExpandableDocumentCommand{\convertlen}{ O{cm} m }
 {
  \dim_to_unit:nn { #2 } { 1 #1 } cm
 }
\ExplSyntaxOff

%% For lattice figure
% Set the overall layout of the tree
\tikzstyle{level 1}=[level distance=3.0cm, sibling distance=0.6cm]
\tikzstyle{level 2}=[level distance=3.5cm, sibling distance=0.6cm]
\tikzstyle{level 3}=[level distance=3.5cm, sibling distance=0.6cm]

% Define styles for bags and leafs
\tikzstyle{l1} = [rectangle, text width=5em, text centered]
\tikzstyle{l2} = [rectangle, text width=5em, text centered]
\tikzstyle{l3} = [rectangle, text width=5em, text centered]

% only when using asmthm
%\newtheorem{definition}{Definition}
%\newtheorem{theorem}{Theorem}

\author{Micky Faas \and Matthijs van Leeuwen}
\title{VOUW: Geometric Pattern Mining using the MDL Principle}
\institute{Leiden Institute for Advances Computer Science}
\begin{document}

\section{Related Work}

As the first pattern mining approach using the MDL principle, Krimp \cite{krimp} was one of the main sources of inspiration for this paper. Many papers on pattern-based modelling using MDL have appeared since, both improving search, e.g., Slim \cite{slim}, and extensions to other problems, e.g., Classy \cite{classy} for rule-based classification.

The problem closest to ours is probably that of geometric tiling, as introduced by Gionis et al.\ \cite{gionis2004tiles} and later also combined with the MDL principle by Tatti and Vreeken \cite{tatti2012stijl}. Geometric tiling, however, is limited to Boolean data and rectangularly shaped patterns (tiles); we strongly relax both these limitations (but as of yet do not support patterns based on densities or noisy occurrences).

Campana et al. \cite{campana2010compression} also use matrix-like input data (textures) and develop a compression-based similarity measure. Their method, however, cannot be used for \emph{explanatory} data analysis as it relies on a generic image compression algorithm that is essentially a black box.

%Geometric pattern mining is different from graph mining, as a matrix is more rigid and each element has a fixed degree of connectedness/adjacency. It is also unrelated to linear algebra, other then using the term `matrix' and a comparable style of notation. 

\end{document}
